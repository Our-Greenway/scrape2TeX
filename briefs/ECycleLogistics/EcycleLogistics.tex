\documentclass[letter]{ourGreenwayBrand}

\headerlabel{Research Brief}

\titletext{Connecting With Cargo Bikes}
\subtitletext{}
\authortext{Serena Sonnenberg}
\editedtext{Jack Lawson And Darnel Harris}
\datetext{August 16, 2025}

\begin{document}
\MakeBrandTitle

Moving goods from one place to another is an essential part of the rhythm of human life.

However, moving urban freight is only becoming more complex due to industry inefficiencies and contributors like traffic congestion, global warming, and noise pollution. According to data fromEnvironment and Climate Change Canada\footnote{\url{https://unfccc.int/documents/194925}}andNatural Resources Canada (http://oee.nrcan.gc.ca/corporate/statistics/neud/dpa/showTable.cfm?type=CP\&sector=tran\&juris=ca\&rn=8\&page=0), Transportation accounts for 24\% of total emissions in Canada. Freight trucks account for 36\% of this figure. Furthermore the share of greenhouse emissions that freight vehicles emit isprojected to surpass (https://unfccc.int/sites/default/files/resource/Canada\%E2\%80\%99s\%20Fourth\%20Biennial\%20Report\%20on\%20Climate\%20Change\%202019.pdf)that of passenger vehicles by 2030.

With a rapidly growing population and increasing reliance on e-commerce and goods delivery services, how might we reimagine urban freight logistics?

\section{The future of freight}
Cargo bikes play a key role in supporting a cleaner and more efficient future for the movement of goods across cities.

When it comes to cargo bikes, modularity and flexibility are king. They come in many shapes and sizes, from utility bikes that are designed like a conventional bike but have a higher load capacity to longtails with a longer rear rack for more carrying space. There are also longjohns with a spacious box between the handlebars and the front wheel, trikes or quadricycles with trailers, boxes, or platforms, or even cargo bikes with a hard shell and windshield to protect drivers and cargo from the elements.

Cargo cycles can be configured to carry goods, people, or both. Some can even handle up to 400 kg of total weight, and many models are available with powerful electric assist motors, most often providing assistance up to 25 km/h, that can make their operation more smooth and accessible.

\section{Solving the last mile}
Cargo bikes are also very helpful for last-mile deliveries via a robust hub model. Electronic vehicle micromobility cargo solutions have been widely adopted by businesses inEurope\footnote{\url{http://www.cyclelogistics.eu/index.php/market-size}}, and more recently have begun to be explored in someCanadian\footnote{\url{https://www.pembina.org/pub/cyclelogistics}}and US cities. In dense urban settings,studies (https://www.researchgate.net/publication/332818991\_Travel\_Time\_Differences\_between\_Cargo\_Cycles\_and\_Cars\_in\_Commercial\_Transport\_Operations)have shown that cargo bikes can be faster than large freight vehicles.

Connections can be made between road freight networks and cycle logistics networks by creatingcargo bike hubs\footnote{\url{http://cyclelogistics.eu/downloads/source-material/planning-cargo-bike-hubs}}, which are small, centrally located consolidation centres. This principle is very similar to a standard distribution hub model, but varies in scale and prioritizes localized reach rather than city-wide distribution. For example, large freight vehicles can drop things off at night and cargo cycles can pick up loads for the last mile. By taking some of these last-mile freight vehicles off of the road, communities will also see an increased quality of life with less noise pollution, air pollution, and traffic. These are all components that interest the Canadian government through both the Transportation 2030 Strategic Plan and higher ordinance planning documentation on complete, transit-oriented communities.

\section{The community conclusion}
Beyond their use for last-mile logistics, cargo bikes can also be a powerful tool for small businesses, organizations, and entrepreneurs. The COVID-19 pandemic and subsequent city shutdown has been a significant challenge for many businesses around Toronto. Even more importantly, COVID-19 has led to a culture shift in consumer habits where local delivery options through Skip the Dishes, Uber Eats, and the like have shifted goal posts for small business.

Our Greenway’s proposed mobility greenway through Northwest Toronto would connect the community and local businesses in a way that the current automobile-centric roads do not. This in fact speaks to a broader issue in Northwest Toronto where decaying public transportation infrastructure have forced residents into a heavily car dependent climate. Northwest Toronto also has a significant amount of industrial freight, high rates of diabetes and cardiovascular disease, and the growing threat of both the gray wave and long COVID.

Micromobility solutions like cargo bicycles can help provide innovative and environmentally friendly logistics solutions that can also improve health outcomes. The future may continue to bring unforeseen challenges, but community connection and sustainable technology such as cargo bikes can help us be ready for what comes next.

Assmann, T., Müller, F., Bobeth, S., Baum, L. (2019).Planning of Cargo Bike Hubs.Cyclelogistics.http://cyclelogistics.eu/downloads/source-material/planning-cargo-bike-hubs\footnote{\url{http://cyclelogistics.eu/downloads/source-material/planning-cargo-bike-hubs}}

Behrensen, A., Sumer, A. (2020).First European Cargo Bike Industry Survey: Results. Cyclelogistics.http://www.cyclelogistics.eu/index.php/market-size\footnote{\url{http://www.cyclelogistics.eu/index.php/market-size}}

Environment and Climate Change Canada, 2019 National Inventory Report 1990–2017: Greenhouse Gas Sources and Sinks in Canada, Part 3, Table A10-2.https://unfccc.int/documents/194925\footnote{\url{https://unfccc.int/documents/194925}}

Government of Canada, Canada’s Fourth Biennial Report on Climate Change (2019), Annex 2.https://unfccc.int/sites/default/files/resource/Canada\%E2\%80\%99s\%20Fourth\%20Biennial\%20Report\%20on\%20Climate\%20Change\%202019.pdf (https://unfccc.int/sites/default/files/resource/Canada\%E2\%80\%99s\%20Fourth\%20Biennial\%20Report\%20on\%20Climate\%20Change\%202019.pdf)

Gruber, J., Narayanan, S. (2019).Travel Time differences between Cargo Cycles and Cars in Commercial Transport Operations.Transportation Research Record: Journal of the Transportation Research Board.https://www.researchgate.net/publication/332818991\_Travel\_Time\_Differences\_between\_Cargo\_Cycles\_and\_Cars\_in\_Commercial\_Transport\_Operations (https://www.researchgate.net/publication/332818991\_Travel\_Time\_Differences\_between\_Cargo\_Cycles\_and\_Cars\_in\_Commercial\_Transport\_Operations)

Natural Resources Canada, “Table 8: GHG Emissions by Transportation Mode,” Comprehensive Energy Use Database.http://oee.nrcan.gc.ca/corporate/statistics/neud/dpa/showTable.cfm?type=CP\&sector=tran\&juris=ca\&rn=8\&page=0 (http://oee.nrcan.gc.ca/corporate/statistics/neud/dpa/showTable.cfm?type=CP\&sector=tran\&juris=ca\&rn=8\&page=0)

\vspace{2em}
\fbox{\parbox{\dimexpr\textwidth-2\fboxsep-2\fboxrule\relax}{
\raggedright
  \small This PDF was automatically generated using the Python-to-LaTeX tool available at~\url{https://github.com/Our-Greenway/scrape2TeX}.\\[0.5em]
  If there are any differences, the online version at~\url{https://www.ourgreenway.ca/e-cycle-logistics} shall prevail.
}}
\end{document}