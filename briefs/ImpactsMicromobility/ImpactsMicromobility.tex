\documentclass[letter]{ourGreenwayBrand}

\headerlabel{Research Brief}

\titletext{The Impacts Of Micromobility}
\subtitletext{}
\authortext{Chantal Lee}
\editedtext{Jack Lawson And Darnel Harris}
\datetext{August 16, 2025}

\begin{document}
\MakeBrandTitle

In an era of climate change, both the transportation and energy industry are grappling with an over-reliance on fossil fuels. Vehicle emissions are one of the leading causes of our warming climate and severe weather, and as it stands many cities are moving slowly to source and construct alternative solutions. According to data fromEnvironment and Climate Change Canada\footnote{\url{https://unfccc.int/documents/194925}}andNatural Resources Canada (http://oee.nrcan.gc.ca/corporate/statistics/neud/dpa/showTable.cfm?type=CP\&sector=tran\&juris=ca\&rn=8\&page=0), transportation accounts for 24\% of total emissions in Canada, and of this, freight trucks account for 36\%. Not only this, but the share of greenhouse emissions that freight vehicles emit isprojected to surpass (https://unfccc.int/sites/default/files/resource/Canada\%E2\%80\%99s\%20Fourth\%20Biennial\%20Report\%20on\%20Climate\%20Change\%202019.pdf)that of passenger vehicles by 2030.

Our Greenway Conservancy (OGC) aims to meet both rising climate-based inequity, and meet threats to our environment by shifting the conversation towards local, purpose-built, green solutions. At the top of this list for Our Greenway are practical micromobility tools.

Micromobility can cover everything from scooters to bikes, electronically assisted or otherwise. Of special note are carrier cycles and cargo cycles. These are devices that can not only move people locally, but also aid businesses in reducing overhead transportation costs as well. Cleaner air and open roads are possible, provided we act quickly to establish strong alternatives to the current culture of fossil fuel use. Examples of micromobility vehicles include traditional bicycles, scooters and skateboards, as well as electrified modes of transportation such as e-bikes, e-scooters and e-skateboards (International Transport Forum, 2020). In Canada, however, we are in part limited by a lack of legal clarity around what constitutes a safe micromobility device. For example, old Federal definitions of e-cycles limit them to three or fewer wheels, which is a huge problem. More wheels would enhance stability, narrow front-facing profiles for door access, and allow for wider scales of use.

There are many social, economic and environmental benefits of micromobility that make it a sustainable and equitable mode of transportation. Micromobility offerings can not only soothe social isolation among seniors through easy access to local green space and amenities, but are aligned with Canada’s fossil fuel reduction and natural resource conservation strategies. High level Canadian city planning documentation, such as theTransportation 2030 Strategic Plan\footnote{\url{https://tc.canada.ca/en/initiatives/transportation-2030-strategic-plan-future-transportation-canada}}, prize innovative green-focused transportation options.  In particular the themes of  bothSafer TransportationandGreen and Innovative Transportationare covered. Both reference electric vehicles or alternative fuel as primary candidates for meeting each thematic goal.

\section{Micromobility benefits communities}
Micromobility offers options to those who can’t drive; such as seniors, those in secondary school or lower, and those living with disabilities. Those with limited mobility options can use microvehicles as a form of transportation through either riding as a passenger or using electric options. At a policy level, micromobility infrastructure provides a solution to the‘last mile’ issues plaguing many public transit networks. In short, transportation costs tend to skyrocket the closer goods become to their destination.

Low-income areas are especially susceptible to this problem due to a lack of public investment in transit systems, which leads to infrequent service and a poorly connected network. In late 2021 OGC commissioned IV Consulting to examine the implementation of a carrier cycle library at Jane and Finch. All told, IV concluded using carrier cycles rather than public transit could save commuters between 30 and 90 minutes depending on the trip.

The greatest barrier to implementing micro-mobility offerings is the lack of dedicated cycle infrastructure in the area. This is most clearly shown in the commuter tendencies for Jane and Finch, which overwhelmingly favour automobiles. With that said, it is entirely possible to change commuter behaviour through focused micromobility offerings. For instance, a pilot study conducted by the San Francisco Transportation Agency showed that, “34\% of respondents used scooters to get to or from public transportation (Said, 2019).” This aligns with IV Consulting’s findings as well, specifically that micromobility offerings such as carrier cycles are a strong supplement to public transit offerings.

\section{Business-based benefits and the local economic advantage}
Micromobility is a swiss army knife that covers everything from a trip to the grocery store to transporting goods. E-cycles and cargo bikes offer economic and personal benefits for users, cities, and businesses. For citizens, this is accomplished by providing greater access amenities, social spaces, and job opportunities via low-cost transit. Employers benefit as well. Micro-mobility options like e-cargo cycles can provide greater access to different neighbourhoods and increase foot traffic in commercial areas to further stimulate local economies. Some cargo cycles can also offer businesses branding or partnership opportunities through decals or community artwork placed directly on the bike itself.

In 2016, the City of Toronto launched a bike lane pilot project along Bloor Street that stretched 2.4km from Avenue Road to Shaw Street, passing through the Annex neighbourhood and Koreatown (Toronto Centre for Active Transportation, 2019). The results from the study showed that businesses reported greater earnings and more customer visits after the launch of the pilot project.

Businesses can also make use of microvehicles such as cargo bikes for the delivery of goods. For instance, local breweries could deliver their product to local bars through e-cargo connections along a greenway. Delivery couriers such as Fedex, Amazon and UPS have added cargo bikes to their fleet due to its affordability and time-efficiency.  As it stands, Our Greenway is one of the only organizations in Toronto that has a library of over a dozen bikes and trailers spread across multiple designs including sub categories like cargo, trishaw, and commuting.

In May of 2022 the price of gas rose to over \$2.00 per litre, a record breaking high in Canada. The price of diesel and gasoline for delivery vans can fluctuate wildly over time whereas micro-vehicles eliminate this cost risk almost entirely. In addition, using micro-vehicles for the delivery of goods can be more time-efficient compared to delivery trucks. Micro-mobility offerings can avoid high traffic areas through using bike lanes, alternative pathways, or dedicated infrastructure developments. For example, FedEx’s Bullitt cargo bikes deliver 25-30\% more parcels compared to FedEx’s vans in downtown areas according to Hans Fogh, the CEO of Bullitt (Cycle Toronto, 2020). This is just one example of micro-mobility solutions fitting into existing logistics frameworks.

\section{Environmental action through local movement}
In an era of intense climate change micromobility options offer environmental benefits that can help fill the gap as cities struggle to adapt existing policies to meet new needs.

Cars and long-haul transportation produce air pollutants, increase greenhouse gas emissions, and lead to respiratory illnesses. Across Canada, the transportation sector accounts for 25\% of greenhouse gas emissions (Natural Resources Canada, 2019). Using microvehicles in place of automobiles can decrease emissions through eliminating tailpipe exhaust entirely. Likewise, it can also help insulate fleet operators from unpredictable fuel costs or price gouging.

Cars powered by petrol or diesel engines produce twenty times more greenhouse gas per unit compared to electric bikes, according to a study conducted in New Zealand (Elliot et. al, 2018). The greenhouse gas reductions from using electric bikes in place of cars powered by petrol or diesel may be significantly higher in Toronto due to Ontario’s clean energy mix.

Therefore, switching modes of transportation from cars powered by petrol or diesel engines to electric bikes can have a significant impact on reducing greenhouse gas emissions and improving public health.

\section{Conclusion? sustainability is practical and profitable.}
Our Greenway Conservancy aims to facilitate opportunities to showcase the powerful social, economic and environmental benefits of micromobility. A major aspect of this is showing residents, businesses and decision makers the clear benefits of micromobility solutions. As such, running a cycle library helps to prove the concept and expand the way we talk about transportation.

With that said, strong partners are essential to delivering on the potential of micromobility offerings. For example, our Conservancy’s Cycling Without Age chapter is gathering data on how carrier cycles can serve seniors and soothe social isolation post COVID-19. The lesson for organizations - not for profit or otherwise - is that broad based advocacy networks armed with practical solutions are essential to drive change.

\newpage
\section{Sources}

\hspace{1em}Elliot et. al (2018)

\hspace{1em} \url{https://reader.elsevier.com/reader/sd/pii/S2352550918301428?token=942048B3E6DD19466D4CFF7FF74ED6E1541E86FFDDA5A5CCE719C4898EC726F01CB8ECC3E9C1BB22C73C6BF1A5D148CB}

\hspace{1em}International Transport Forum (2020)

\hspace{1em}\url{https://www.itf-oecd.org/sites/default/files/docs/safe-micromobility_1.pdf}

\hspace{1em}Said (2019)

\hspace{1em}\url{https://www.sfchronicle.com/business/article/SF-finds-controlled-scooter-pilot-a-success-13764061.php?fbclid=IwAR1OWRb31WXx1XcmtpUy2KpWNKFgL2zMBmVZ6aXgPs8QftCLvvxaaRYOs94}

\hspace{1em}Toronto Centre for Active Transportation (2019)

\hspace{1em}\url{https://www.tcat.ca/wp-content/uploads/2017/12/Bloor-Economic-Impact-Study-Full-Report-2019-09-03.pdf}

\hspace{1em}Cycle Toronto (2020)

\hspace{1em} \url{https://www.cycleto.ca/news/benefits-cargo-bike-parcel-delivery}

\hspace{1em}Natural Resources Canada (2019)

\hspace{1em}\url{https://www.nrcan.gc.ca/sites/www.nrcan.gc.ca/files/oee/pdf/transportation/alternative-fuels/resources/pdf/NRCan_NGRoadmap_e_WEB.pdf}

\vspace{2em}
\fbox{\parbox{\dimexpr\textwidth-2\fboxsep-2\fboxrule\relax}{
  \small This PDF was automatically generated using the Python-to-LaTeX tool available at~\url{https://github.com/Our-Greenway/scrape2TeX}.\\[0.5em]
  If there are any differences, the online version located at~\url{https://www.ourgreenway.ca/impactsofmicromobility} shall prevail.
}}
\end{document}