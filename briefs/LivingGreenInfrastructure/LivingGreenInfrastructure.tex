\documentclass[letter]{ourGreenwayBrand}

\headerlabel{Research Brief}

\titletext{A Living Greenway In Northwest Toronto}
\subtitletext{}
\authortext{Serena Sonnenberg}
\editedtext{Jack Lawson And Darnel Harris}
\datetext{August 16, 2025}

\begin{document}
\MakeBrandTitle

Incorporating nature into the city brings beauty and comfort. But living infrastructure also enhances the quality of life for both structures and citizens. Our Greenway’s mission is in part to develop a network of rain gardens that run alongside the length of the proposed mobility network. On the surface, these gardens are a great way to make greenways more beautiful and provide a natural barrier to vehicular traffic. The full impacts, however, run much deeper than that.

\section{Concrete and grass: moving away from impermeable surfaces}
Currently many areas in Northwest Toronto are heavily paved with impervious materials such as concrete and asphalt.

According toCredit Valley Conservation (https://cvc.ca/wp-content/uploads/2016/11/DISCUSSION-PAPER-Roads-and-Runoff-Implementing-Green-Streets-in-the-Greater-Golden-Horseshoe.pdf), stormwater runoff in urban areas with large amounts of impervious cover carries automotive pollution, sediment, road salt, and other environmental contaminants through storm drains and directly into local bodies of water. This disrupts the natural water balance, and adversely impacts sensitive species.  Climate change is already bringing an increase of frequent, unpredictable and severe storm events that canraise the risk of flooding\footnote{\url{https://trca.ca/conservation/flood-risk-management/understand/}}. In an urban context, flooding is caused by large levels of stormwater runoff that overwhelm the drainage systems. When a sanitation system is backed up during a storm past the overflow reservoirs this leads to contaminated water flooding streets, basements, and low-lying areas.

Our Greenway’s rain garden proposal is designed to support plants that are resilient to the harsh, salty conditions of the roadside. This can include staghorn sumac, cottonwood, and native grasses and wildflowers in addition to many members of the succulent family. These planted areas serve to significantly reduce levels of stormwater runoff, and improve the water quality by allowing water to slowly drain into the earth and filtering it as it runs back into Lake Ontario.

\section{Once around the rain garden: case studies in a north american context}
Similar Low Impact Development (LID) projects carried out by Credit Valley Conservation such as aroad right-of-way bioretention project in Lakeview (https://cvc.ca/wp-content/uploads/2014/04/Lakeview-Case-Study\_Apr\_04\_2014\_FINAL1.pdf)have demonstrated a25\% reduction\footnote{\url{https://cvc.ca/wp-content/uploads/2012/11/Lakeview-Monitoring-Factsheet.pdf}} in road resurfacing costs due to improvements in stormwater management and a92\% overall reduction (http://www.creditvalleyca.ca/wp-content/uploads/2016/06/TechReport\_Lakeview\_Final.pdf)in runoff volume (compared to 68\% reduction in runoff observed with ordinary grass swales).

Similarly, the bioswales that line theIndy Cultural Trail\footnote{\url{https://indyculturaltrail.org/2015/06/02/bioswales-sustainability-along-the-trail/}}in Indianapolis, Indiana divert an impressive 4 million gallons of rainwater from the city’s sewage system per year. This takes some stress off of the city’s stormwater drainage system, reduces contamination of local water bodies, and reduces the risk of flooding during severe weather events.

The City of Toronto’sOfficial Plan Amendment 262\footnote{\url{http://app.toronto.ca/tmmis/viewAgendaItemHistory.do?item=2015.PG7.2}}revised Policy 3.4.19 of the Official Plan to “articulate innovative methods of stormwater management including stormwater attenuation and re-use and use of green infrastructure. ”This amendment continues to be in full force, and demonstrates that living green infrastructure projects such as the Greenway’s rain gardens support the city’s long term goals of sustainability in the face of climate change.

\section{A culture of custodianship}
With that said bioretention gardens require maintenance with both members of City leadership, and the Communities they service.

According to amemorandum (https://www.epa.gov/sites/production/files/2016-11/documents/final\_gi\_maintenance\_508.pdf)released by the United States EPA, key points of maintenance include caring for plants, managing weeds, controlling the accumulation of sediment, trash or organic material, keeping inlets and outlets unclogged, and making sure that excessive erosion does not occur. This regular maintenance is important, but is not different from other landscaped or turfed areas, and does not require specialized equipment. Some forms of maintenance, such as weeding and debris removal, can be done in semi-annual community clean-up days, as is the case at the Indy Cultural Trail. This also builds a relationship with the space through regular community interaction, especially when supported by local Council members and staff.

It is also important to monitor the plant life within the rain garden, to observe what thrives within the roadside conditions, what does not, and whether any substitutions need to be made in subsequent seasons given the lessons learned through observation.

\section{Conclusion}
Rain gardens are a beautiful feature that brings an abundance of colour, character, joy, and protection to communities of all stripes. They provide a much needed urban oasis filled with native plants that support local biodiversity and provide a physical buffer between greenway users and vehicles. But they are also cost effective and highly functional pieces of living infrastructure that contribute toward a healthier and more sustainable Toronto.

\newpage
\section{Sources}

\hspace{1em}Blakelock, C., \& Maynes, C. (2015). (publication).Roads and Runoff: Implementing Green Streets in the Greater Golden Horseshoe. Port Credit, Ontario: Credit Valley Conservation.\url{https://cvc.ca/wp-content/uploads/2016/11/DISCUSSION-PAPER-Roads-and-Runoff-Implementing-Green-Streets-in-the-Greater-Golden-Horseshoe.pdf} \url{https://cvc.ca/wp-content/uploads/2016/11/DISCUSSION-PAPER-Roads-and-Runoff-Implementing-Green-Streets-in-the-Greater-Golden-Horseshoe.pdf}

\hspace{1em}Toronto and Region Conservation Authority.Understand - Flood Risk Management. Toronto and Region Conservation Authority (TRCA).\url{https://trca.ca/conservation/flood-risk-management/understand/} \url{https://trca.ca/conservation/flood-risk-management/understand/.}

\hspace{1em}Credit Valley Conservation (2014). (publication).Lakeview Neighbourhood Case Study.Mississauga, Ontario: Credit Valley Conservation.https://cvc.ca/wp-content/uploads/2014/04/Lakeview-Case-Study\textbackslash{}textbackslash\{\}\_Apr\textbackslash{}textbackslash\{\}\_04\textbackslash{}textbackslash\{\}\_2014\textbackslash{}textbackslash\{\}\_FINAL1.pdf \url{https://cvc.ca/wp-content/uploads/2014/04/Lakeview-Case-Study_Apr_04_2014_FINAL1.pdf}

\hspace{1em}Credit Valley Conservation (2012). (publication).Lakeview Monitoring Factsheet.Mississauga, Ontario: Credit Valley Conservation.\url{https://cvc.ca/wp-content/uploads/2012/11/Lakeview-Monitoring-Factsheet.pdf} \url{https://cvc.ca/wp-content/uploads/2012/11/Lakeview-Monitoring-Factsheet.pdf}

\hspace{1em}Frey, S. (2015, June 2).Bioswales: Sustainability Along the Cultural Trail. Indianapolis Cultural Trail.\url{https://indyculturaltrail.org/2015/06/02/bioswales-sustainability-along-the-trail/} \url{https://indyculturaltrail.org/2015/06/02/bioswales-sustainability-along-the-trail/.}

\hspace{1em}Toronto City Coucil (2015).Official Plan Five Year Review: Final Recommendation Report - Amendments to the Official Plan Environmental Policies and Designation of Environmentally Significant Areas.City of Toronto.\url{http://app.toronto.ca/tmmis/viewAgendaItemHistory.do?item=2015.PG7.2} \url{http://app.toronto.ca/tmmis/viewAgendaItemHistory.do?item=2015.PG7.2}

\hspace{1em}Tetra Tech, Inc. (2016). (technical memorandum).Operation and Maintenance of Green Infrastructure Receiving Runoff from Roads and Parking Lots.United States Environmental Protection Agency.https://www.epa.gov/sites/production/files/2016-11/documents/final\textbackslash{}textbackslash\{\}\_gi\textbackslash{}textbackslash\{\}\_maintenance\textbackslash{}textbackslash\{\}\_508.pdf \url{https://www.epa.gov/sites/production/files/2016-11/documents/final_gi_maintenance_508.pdf}

\vspace{2em}
\fbox{\parbox{\dimexpr\textwidth-2\fboxsep-2\fboxrule\relax}{
\raggedright
  \small This PDF was automatically generated using the Python-to-LaTeX tool available at~\url{https://github.com/Our-Greenway/scrape2TeX}.\\[0.5em]
  If there are any differences, the online version at~\url{https://www.ourgreenway.ca/living-green-infrastructure} shall prevail.
}}
\end{document}